\documentclass{article}
\usepackage[backend=biber,citestyle=ieee]{biblatex}
\usepackage[english]{babel}
% \usepackage[swedish]{babel}
\usepackage{graphicx}
\usepackage{csquotes}
\usepackage{float}
\usepackage{datetime}
\usepackage[title]{appendix}
% \usepackage{a4wide} %For wider content on page
% \usepackage{amsmath} %For multiline equations 

\usepackage{fancyhdr}   %page header
\pagestyle{fancy}

% \usepackage[parfill]{parskip} %Line skip between paragraphs instead of indent

\usepackage{xcolor}
\usepackage{listings}

\definecolor{codegreen}{rgb}{0,0.6,0}
\definecolor{codegray}{rgb}{0.5,0.5,0.5}
\definecolor{codepurple}{rgb}{0.58,0,0.82}
\definecolor{backcolour}{rgb}{0.95,0.95,0.95}
\lstdefinestyle{mystyle}{
    backgroundcolor=\color{backcolour},   
    commentstyle=\color{codegreen},
    keywordstyle=\color{magenta},
    numberstyle=\tiny\color{codegray},
    stringstyle=\color{codepurple},
    basicstyle=\ttfamily\footnotesize,
    breakatwhitespace=false,         
    breaklines=true,                 
    captionpos=b,                    
    keepspaces=false,                 
    numbers=left,                    
    numbersep=5pt,                  
    showspaces=false,                
    showstringspaces=false,
    showtabs=false,                  
    tabsize=1
}
\lstset{style=mystyle}

%\addbibresource{sources.bib}

\newcommand{\getauthor}{Elias Berglin} %Author
\newcommand{\gettitle}{COAP implementation} %Title

\newdateformat{daymonthyear}{\ordinal{DAY} \monthname[\THEMONTH] \THEYEAR} %Date

\title{\gettitle}
\author{\getauthor}

\date{\daymonthyear\today} %Remove for swedish date

\begin{document}

    % Title 
    \pagenumbering{gobble}
    \maketitle
    \newpage

    % Page header and footer
    \pagenumbering{arabic}
    \fancyhf{}
    \lhead{\getauthor}
    \rhead{\gettitle}
    \rfoot \thepage

    % Document starts here
    \section{Creating a message}
    I created four different functions for the different message types GET, POST, PUT, DELETE.
    Each function takes a URI path and POST and PUT also take a payload as a string. It returns a byte array
    this array is the finished message. 
    \begin{lstlisting}[language=go]
func createPut(path string, payload string) []byte
    \end{lstlisting}
    The first part of every function is the creation of the header.
    The process of creating the header is always the same for the given method.
    Here we can see the creation process of a POST request:
    \begin{lstlisting} [language=go]
ret := make([]byte, 0)
firstByte := byte(0b01010000)
ret = append(ret, firstByte)
code := byte(0x02)
ret = append(ret, code)
r := rand.Uint32()
id := make([]byte, 4)
binary.BigEndian.PutUint32(id, r)
id = id[:2]
ret = append(ret, id...)
    \end{lstlisting}
    The variable ret is the byte array that is returned. 
    first byte creates the first byte of the message containing the version, message type and token length
    In this case we have version 1, type 1 and token length 0.
    I then append the byte to ret.

    The variable code is next containing the method. This gets a value of 2 that is a POST message.
    This is then appended to the array.

    The message id is always randomized. The smallest unsigned int that can be randomized in Go is 32bits
    long and thus is 4 bytes. So, I first create a 4-byte long array and put the integer there and last, I remove the 2 last bytes
    creating a 2 byte long random int that is then appended to the array.
    As I do not include a token this is skipped

    Next part is creating options. There are two options that is required first is the URI and then the content type. I created a separate function for this.

\begin{lstlisting}[language=go]
func createOption(name string, data []byte, lastDelta int) ([]byte, int) {
    switch name {
	    case "uri":
		    delta := (11 - lastDelta) << 4
		    lastDelta = 11 - lastDelta
		    l := len(data)
		    header := delta + l
		    return append([]byte{byte(header)}, data...), lastDelta
	    case "contentType":
		    delta := (12 - lastDelta) << 4
		    lastDelta = 12 - lastDelta
		    l := len(data)
		    header := delta + l
		    return append([]byte{byte(header)}, data...), lastDelta
	}
	return nil, 0
}
\end{lstlisting}
This function first takes the name of the option. Then it takes the option data as a byte array
and lastly it takes the last delta from the previous option. I then have a switch case statement
for creating the different options depending in the option name I took as input.
The function returns the finished option as an array and an integer representing the last delta.
The first thing I do is calculate the option delta and bit shift it 4 times to move the 4 bits to the last 4 bits
I then get the length of the option data and add that to the delta
finally, I return a byte array with the delta and option byte followed by the option data and the calculated delta.

After the options are returned, they are appended to the message array in creation function
for the message. Here is an example from the POST request:
\begin{lstlisting}[language=go]
lastDelta := 0
option, lastDelta := createOption("uri", []byte(path), lastDelta)
ret = append(ret, option...)
option, _ = createOption("contentType", make([]byte, 0), lastDelta)
ret = append(ret, option...)
\end{lstlisting}
Here we can see the user input as both path and payload. They are a string but here I convert
them to byte arrays before sending them in to the createOption function. You can also see
that the last delta starts at 0 and is overwritten by the returning int from the function.

Lastly, I append the delimiter and as we can see in the following code snippet it is FF in
hexa decimal and after that I append the payload thus is determined by the user input
\begin{lstlisting}[language=go]
    ret = append(ret, byte(0xFF))
	ret = append(ret, []byte(payload)...)
\end{lstlisting}
The other functions for GET, PUT and DELETE are similar, the difference is the method bits in the header
and the fact that option and delete does not contain a payload.
\section{Parse a message}
To parse the message, I use two functions. One for parsing the message and one for parsing the options.
For the main parsing function, I receive a byte array containing the message and the number of
bytes it contains. I return an instance of a COAP message struct.
\begin{lstlisting}[language=go]
	message := COAPMessage{
		Version:   int((arr[0] & 0b11000000) >> 6),
		T:         int((arr[0] & 0b00110000) >> 4),
		TKL:       int(arr[0] & 0b00001111),
		Code:      int(arr[1]),
		MessageID: int(binary.BigEndian.Uint16([]byte{arr[2], arr[3]})),
	}
\end{lstlisting}
First, I create an instance of the message and assign the fixed values. I do this
by doing an and on the byte with a bit string containing ones on the bits
where the data I am interested in getting. I then shift the resulting bits to place
them so their value is correct.

When the set bits are done, I then try to find the delimiter byte (FF in hex).
I store the value so I can take out the subset of bits containing the options.
\begin{lstlisting}[language=go]
var index int
for i, b := range arr {
	if b == 0xFF {
		index = i
	}
}
if index == 0 {
	index = n
}
 \end{lstlisting}
 If I do not find the delimiter, I set the index of the final bit of the options to
 the end of the whole message because I know the rest of the message is only options.

 If token length is 0 then I set the start of options to the fifth byte otherwise
 I set it to the byte after token. I then send the subset of the array to a function
 that decodes the options. In that function I start the delta on 0 and add to it with
 every option and then save that to a struct representing an option. For now
 I do nothing with the option data and just save as a byte array. 
 \section{Printing the message}
 The printing of a message is done by a void function taking an instance of the COAP
 message class. 
\begin{lstlisting}[language=go]
func printCOAP(c COAPMessage) {
	fmt.Println("Version:", c.Version, "Message Type:", c.T, "Token leangth:", c.TKL)
	fmt.Println("Metod/Response:", "\""+parseMethodCode(c.Code)+"\"", "Message id:", c.MessageID)
	fmt.Println("Token:", c.Token)
	fmt.Println("Options:")
	c.Format = printOptions(c.Options)
	if c.Format != "text/plain" {
		fmt.Println("FIX FORMAT")
	} else {
		fmt.Println("Payload:\n", "\t"+string(c.Payload))
	}
}
\end{lstlisting}
Helper functions help the main function handle options and converting identification
number to human readable strings. The complete code can be found in Appendix \ref{appendix:code}
 



    % Sources
    %\newpage
    %\printbibliography

    \newpage
    \begin{appendices}

        \section{Code}
        \label{appendix:code}
        \begin{lstlisting}[language=go]
            package main

import (
	"encoding/binary"
	"fmt"
	"math/rand"
	"net"
	"os"
	"os/exec"
	"strconv"
	"time"
)

//40 01 04 d2 b4 74 65 73 74

type COAPOption struct {
	Delta   int
	Leangth int
	Value   []byte
}

type COAPMessage struct {
	Version   int
	T         int
	TKL       int
	Code      int
	MessageID int
	Token     string
	Options   []COAPOption
	Payload   []byte
	Format    string
}

func parseMethodCode(c int) string {
	switch c {
	case 1:
		return "GET"
	case 2:
		return "POST"
	case 3:
		return "PUT"
	case 4:
		return "DELETE"
	case 65:
		return "Created"
	case 66:
		return "Deleted"
	case 67:
		return "Valid"
	case 68:
		return "Changed"
	case 69:
		return "Continue"
	case 132:
		return "Not Found"
	}

	return "not in list: " + strconv.Itoa(c)
}

func parseOptionCode(c int) (string, string) {
	switch c {
	case 4:
		return "Etag", "opaque"
	case 8:
		return "Location Path", "string"
	case 11:
		return "Uri-path", "string"
	case 12:
		return "Content Format", "string"
	}

	return "Number " + strconv.Itoa(c) + " is not in list", "null"
}

func createOption(name string, data []byte, lastDelta int) ([]byte, int) {
	switch name {
	case "uri":
		delta := (11 - lastDelta) << 4
		lastDelta = 11 - lastDelta
		l := len(data)
		header := delta + l
		return append([]byte{byte(header)}, data...), lastDelta
	case "contentType":
		delta := (12 - lastDelta) << 4
		lastDelta = 12 - lastDelta
		l := len(data)
		header := delta + l
		return append([]byte{byte(header)}, data...), lastDelta
	}

	return nil, 0
}

func parseOptionsHeader(header byte) (int, int) {
	delta := int((header & 0b11110000) >> 4)
	len := int(header & 0b00001111)
	return delta, len
}

func parseOptions(arr []byte) []COAPOption {
	var options []COAPOption
	cursor := 0
	lastDelta := 0
	for cursor < len(arr) {
		delta, len := parseOptionsHeader(arr[cursor])
		lastDelta += delta
		cursor++
		var val []byte
		if len != 0 {
			val = arr[cursor : cursor+len]
		}
		temp := COAPOption{
			Delta:   lastDelta,
			Leangth: len,
			Value:   val,
		}
		options = append(options, temp)
		cursor += len

	}
	return options
}

func parseMessage(arr []byte, n int) COAPMessage {
	message := COAPMessage{
		Version:   int((arr[0] & 0b11000000) >> 6),
		T:         int((arr[0] & 0b00110000) >> 4),
		TKL:       int(arr[0] & 0b00001111),
		Code:      int(arr[1]),
		MessageID: int(binary.BigEndian.Uint16([]byte{arr[2], arr[3]})),
	}
	var index int
	for i, b := range arr {
		if b == 0xFF {
			index = i
		}
	}
	if index == 0 {
		index = n
	}

	optionStart := 4

	if message.TKL != 0 {
		message.Token = string(arr[4 : 4+message.TKL])
		optionStart = 5 + message.TKL
	}

	optionByte := arr[optionStart:index]
	message.Options = parseOptions(optionByte)

	if index != n {
		message.Payload = arr[index+1 : n]
	}

	return message
}

func createGet(path string) []byte {
	var ret []byte
	firstByte := byte(0b01010000)
	ret = append(ret, firstByte)
	code := byte(0x01)
	ret = append(ret, code)
	r := rand.Uint32()
	id := make([]byte, 4)
	binary.BigEndian.PutUint32(id, r)
	id = id[:2]
	ret = append(ret, id...)
	lastDelta := 0
	option, lastDelta := createOption("uri", []byte(path), lastDelta)
	ret = append(ret, option...)
	option, _ = createOption("contentType", make([]byte, 0), lastDelta)
	ret = append(ret, option...)

	return ret
}

func createPost(path string, payload string) []byte {
	ret := make([]byte, 0)
	firstByte := byte(0b01010000)
	ret = append(ret, firstByte)
	code := byte(0x02)
	ret = append(ret, code)
	r := rand.Uint32()
	id := make([]byte, 4)
	binary.BigEndian.PutUint32(id, r)
	id = id[:2]
	ret = append(ret, id...)
	lastDelta := 0
	option, lastDelta := createOption("uri", []byte(path), lastDelta)
	ret = append(ret, option...)
	option, _ = createOption("contentType", make([]byte, 0), lastDelta)
	ret = append(ret, option...)
	ret = append(ret, byte(0xFF))
	ret = append(ret, []byte(payload)...)

	return ret
}

func createPut(path string, payload string) []byte {
	ret := make([]byte, 0)
	firstByte := byte(0b01010000)
	ret = append(ret, firstByte)
	code := byte(0x03)
	ret = append(ret, code)
	r := rand.Uint32()
	id := make([]byte, 4)
	binary.BigEndian.PutUint32(id, r)
	id = id[:2]
	ret = append(ret, id...)
	lastDelta := 0
	option, lastDelta := createOption("uri", []byte(path), lastDelta)
	ret = append(ret, option...)
	option, _ = createOption("contentType", make([]byte, 0), lastDelta)
	ret = append(ret, option...)
	ret = append(ret, byte(0xFF))
	ret = append(ret, []byte(payload)...)

	return ret
}

func createDelete(path string) []byte {
	ret := make([]byte, 0)
	firstByte := byte(0b01010000)
	ret = append(ret, firstByte)
	code := byte(0x04)
	ret = append(ret, code)
	r := rand.Uint32()
	id := make([]byte, 4)
	binary.BigEndian.PutUint32(id, r)
	id = id[:2]
	ret = append(ret, id...)
	lastDelta := 0
	option, lastDelta := createOption("uri", []byte(path), lastDelta)
	ret = append(ret, option...)
	option, _ = createOption("contentType", make([]byte, 0), lastDelta)
	ret = append(ret, option...)

	return ret
}

func printOptions(options []COAPOption) string {
	format := ""
	for _, option := range options {
		optionName, optionFormat := parseOptionCode(option.Delta)
		if optionName == "Content Format" {
			if len(option.Value) == 0 {
				format = "text/plain"
				optionFormat = "CF"
			}
		}
		fmt.Print("\t" + optionName + ": ")
		switch optionFormat {
		case "string":
			fmt.Println(string(option.Value))
		case "CF":
			fmt.Println(format)
		case "opaque":
			fmt.Println(option.Value)
		}

	}
	return format
}

func printCOAP(c COAPMessage) {
	fmt.Println("Version:", c.Version, "Message Type:", c.T, "Token leangth:", c.TKL)
	fmt.Println("Metod/Response:", "\""+parseMethodCode(c.Code)+"\"", "Message id:", c.MessageID)
	fmt.Println("Token:", c.Token)
	fmt.Println("Options:")
	c.Format = printOptions(c.Options)
	if c.Format != "text/plain" {
		fmt.Println("FIX FORMAT")
	} else {
		fmt.Println("Payload:\n", "\t"+string(c.Payload))
	}
}

func sendCOAP(method string) {
	conn, err := net.Dial("udp", "coap.me:5683")
	if err != nil {
		panic(err)
	}
	var msg []byte
	var uri string
	fmt.Println("Type endpoint")
	fmt.Scanln(&uri)
	switch method {
	case "GET":
		msg = createGet(uri)
		break
	case "POST":
		var payload string
		fmt.Println("Type payload")
		fmt.Scanln(&payload)
		msg = createPost(uri, payload)
		break
	case "PUT":
		var payload string
		fmt.Println("Type payload")
		fmt.Scanln(&payload)
		msg = createPut(uri, payload)
		break
	case "DELETE":
		msg = createDelete(uri)
		break
	default:
		msg = make([]byte, 0)
		break
	}

	fmt.Println("Created following Message:")
	fmt.Println("------------------------------------------------")
	printCOAP(parseMessage(msg, len(msg)))
	fmt.Println("------------------------------------------------")
	_, err = conn.Write(msg)
	if err != nil {
		panic(err)
	}
	response := make([]byte, 1024)
	n, err := conn.Read(response)
	if err != nil {
		panic(err)
	}
	response = response[:n]
	fmt.Print("\n")
	fmt.Println("Got following response:")
	fmt.Println("------------------------------------------------")
	printCOAP(parseMessage(response, n))
	fmt.Println("------------------------------------------------")
	fmt.Println("Press any key to continue")
	fmt.Scanln()
	cmd := exec.Command("clear") //Linux example, its tested
	cmd.Stdout = os.Stdout
	cmd.Run()

}

func main() {
	rand.Seed(time.Now().UnixMicro())
	var option string
	run := true
	/* msg := createGet()c
	printCOAP(parseMessage(msg, len(msg))) */
	for run {
		fmt.Println("Select message to send :")
		fmt.Println("1: GET")
		fmt.Println("2: POST")
		fmt.Println("3: PUT")
		fmt.Println("4: DELETE")
		fmt.Println("Any other key to exit")
		fmt.Scanln(&option)

		switch option {
		case "1":
			sendCOAP("GET")
		case "2":
			sendCOAP("POST")
		case "3":
			sendCOAP("PUT")
		case "4":
			sendCOAP("DELETE")
		default:
			run = false
		}
		option = ""
	}
}

        \end{lstlisting}

    \end{appendices}

\end{document}