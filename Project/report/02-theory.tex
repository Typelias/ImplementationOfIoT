\section{Theory}
\label{ch:theory}
\noindent

\subsection{CoAP}
CaAP is a communication protocol over UDP aimed for lower powered computers. It is often used by small microcontrollers that only have limited storage and RAM available. CoAP uses the client server model. The server accepts requests and then responds. The server also provides methods for clients to discover resources on the server. CoAP was designed to be easily translated to HTTP and uses similar request to HTTP, for example GET and POST \cite{rfc7252}.

\subsection{MQTT}
MQTT is a communication protocol over TCP. It is lightweight and provides an open communication channel between devices. The MQTT system consists of one broker and many clients. A client can subscribe 
\subsection{Flutter}


% In the report's theory study, sometimes called Related work, there may be additional facts required for the reader's understanding of the report. At this point a summary of background material in the area should be provided, i.e. standards, scientific articles, books, magazines, documents on the web, technical reports and user manuals. Explain pedagogically with clear examples and many illustrations.

% It should be demonstrated that you have an awareness of the context and the background of your work in addition to that carried out by you within the project. Explain the aim of the technology that you describe, and not only how the technology works. For D-level you should display an awareness of the key research within the area, in order to ensure that your work has a certain news value. However it is vital that you do not deviate too much from your research problem.

% Your assignment is not to write a textbook. It is important to find an appropriate balance between related work and your own results. The theory study should only constitute a minor portion of a thesis.

% Instead of “Theory” or “Related work”, the heading may very well be a specific topic, for example “The GSM standard” or ”A survey on the research field of X".

% If the theoretical study section is rather brief then it is possible to include it within the Introduction chapter.

% If your method is to undertake a critical literature study you normally do not have to have a separate chapter with background material because all sources you refer to are summarized in the results chapter. Your criticism of the sources and the arguments for your personal opinions are thus placed in the concluding chapter.

% \subsection{Definition of terms and abbreviations}
% \label{ch:theory:definitions}
% Terms and abbreviations that are important for the reader's continued understanding are explained in this chapter. The first time you insert text that uses a concept or an abbreviation you should also explain it, even if it is already defined in the terminology section. The concept is typed using the italic style.

% The first time an abbreviation (abbr.) is used it is typed within the parenthesis after its explanation, as illustrated in this sentence.

% \subsubsection{Example of level 3 heading}
% \label{ch:theory:level3-heading}
% Avoid too many heading levels.

% \subsection{To review or quote}
% \label{ch:theory:review:quote}
% You review when you reproduce content using your own words.

% Example: Forslund [4] recommends more informative headings be used in technical reports and that one should, in particular, provide important information in the sub headings.

% You quote when you literally reproduce a phrase, a sentence or paragraph. Quotations under 50 words are to be placed within quotation marks. To quote Strömqvist could be a suitable illustration in this context: “It may be difficult to write, but it is also fun”.

% Quotations over 50 words should be reproduced in the form of block quotations. The text block is centered on the page without quotes and in small caps. The source is stated in direct connection to the block quotation.

% Normally you review instead of using quotes. You can use direct quotations if you wish to reproduce established definitions of concepts, which you believe an author has formulated himself in a particularly suitable manner, when you require aid of an authority, or when you wish to demonstrate that an author is wrong.

% \subsection{References and source references}
% Kindly observe! To reproduce a text without stating its source is to be considered as plagiarism and is thus defined as serious cheating.

% A list of references is placed at the end of the report in order to give the reader overall information regarding all reviewed sources, quotes or for any other reasons that you need to refer to in the text. The sources should be carefully stated so that the reader can check if it is available in libraries or on the internet. Sometimes it might be that verbal sources and other correspondence are included in the source list, but this is unusual in technical reports.

% Refrain from using less trustworthy sources, instead stick to using material written by authorities in the subject matter. Private sites and exam papers are seen as having a low reliability as sources. This is especially true if the exam paper is of a lower level then your own paper.

% Use only sources in the list that you refer to or quote in the continuous text. All sources that are used in the source list should be linked to the report through reference in the continuous text, according to the Vancouver-system, which commonly occurs in reports regarding technical matters.

% According to the Vancouver-system the source list is arranged in the same order as the sources appear in the continuous text, the source reference is to be stated in the text with a figure within square brackets, i.e. or , . They should also be stated in this order in the source list. Examples of source reference: According to Eriksson   dynamic SFNs can provide significant performance improvements.
